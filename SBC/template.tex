\documentclass[12pt]{article}

\usepackage{sbc-template}

\usepackage[brazil]{babel} 
\usepackage[latin1]{inputenc}  

\usepackage{amssymb}
\usepackage{amsmath}
\usepackage{graphicx}
\usepackage[retainorgcmds]{IEEEtrantools}
\usepackage[colorlinks]{hyperref}
\usepackage{rotating}
\usepackage{booktabs}
\usepackage{siunitx}
\usepackage{pdflscape}
% disable labelindent
\let\labelindent\relax
\usepackage{enumitem}
\usepackage{multirow}
\usepackage[normalem]{ulem}
\useunder{\uline}{\ul}{}
\usepackage{setspace} % \doublespacing
% \usepackage[numbers,square]{natbib}
% \usepackage{algorithm, algorithmic}
\usepackage[linesnumbered, ruled, vlined]{algorithm2e}
\usepackage{rotating}
\usepackage{subfigure}

% tikz libraries
\usepackage{tkz-graph}
\usepackage{verbatim}
\usetikzlibrary{arrows,arrows.meta,shapes,decorations.pathmorphing, decorations.markings}
\let\svtikzpicture\tikzpicture
\def\tikzpicture{\noindent\svtikzpicture}

\newtheorem{theorem}{Teorema}[section]
\newtheorem{lemma}[theorem]{Lema}
\newtheorem{proposition}[theorem]{Proposi��o}
\newtheorem{corollary}[theorem]{Corol�rio}

\newenvironment{proof}[1][Prova]{\begin{trivlist}
\item[\hskip \labelsep {\bfseries #1}]}{\end{trivlist}}
\newenvironment{definition}[1][Defini��o]{\begin{trivlist}
\item[\hskip \labelsep {\bfseries #1}]}{\end{trivlist}}
\newenvironment{example}[1][Exemplo]{\begin{trivlist}
\item[\hskip \labelsep {\bfseries #1}]}{\end{trivlist}}
\newenvironment{remark}[1][Observa��o]{\begin{trivlist}
\item[\hskip \labelsep {\bfseries #1}]}{\end{trivlist}}

\newcommand{\qed}{\nobreak \ifvmode \relax \else
      \ifdim\lastskip<1.5em \hskip-\lastskip
      \hskip1.5em plus0em minus0.5em \fi \nobreak
      \vrule height0.75em width0.5em depth0.25em\fi}
           
\sloppy

\title{T�tulo do artigo}

\author{Iago A. Carvalho\inst{1}, Homer J. Simpson\inst{2}}

\address{Departamento de Ci�ncia da Computa��o - Universidade Federal de Minas Gerais \\
Belo Horizonte - MG - Brasil
\nextinstitute
  Twenty Century Fox\\
  Springfield, USA
\email{iagoac@dcc.ufmg.br, homer@thesimpsons.com}
}

\begin{document} 

\maketitle

\begin{abstract}
There will be an abstract here.
\end{abstract}

\begin{resumo} 
Existir� um resumo aqui
\end{resumo}

\section{Introdu��o} \label{sec:intro}

\section*{Agradecimentos}
O presente trabalho foi realizado com apoio da Coordena��o de Aperfei�oamento de Pessoal de N�vel Superior - Brasil (CAPES) - C�digo de Financiamento 001, do Conselho Nacional de Desenvolvimento Cient�fico e Tecnol�gico - Brasil (CNPq) e da Funda��o de Amparo � Pesquisa do Estado de Minas Gerais - Brasil (FAPEMIG).
% This study was financed in part by the \emph{Coordena��o de Aperfei�oamento de Pessoal de N�vel Superior - Brasil} (CAPES) - Finance Code 001, the \emph{Conselho Nacional de Desenvolvimento Cient�fico e Tecnol�gico - Brasil} (CNPq), and the \emph{Funda��o de Amparo � Pesquisa de Minas Gerais - Brasil} (FAPEMIG).

\bibliographystyle{sbc}
\bibliography{bibsample}

\end{document}
