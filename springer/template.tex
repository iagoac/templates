% First comes an example EPS file -- just ignore it and
% proceed on the \documentclass line
% your LaTeX will extract the file if required
% \begin{filecontents*}{example.eps}
%     %!PS-Adobe-3.0 EPSF-3.0
%     %%BoundingBox: 19 19 221 221
%     %%CreationDate: Mon Sep 29 1997
%     %%Creator: programmed by hand (JK)
%     %%EndComments
%     gsave
%     newpath
%     20 20 moveto
%     20 220 lineto
%     220 220 lineto
%     220 20 lineto
%     closepath
%     2 setlinewidth
%     gsave
%     .4 setgray fill
%     grestore
%     stroke
%     grestore
% \end{filecontents*}

%
\RequirePackage{fix-cm}

\documentclass{svjour3}                      % onecolumn (standard format)
%\documentclass[smallcondensed]{svjour3}     % onecolumn (ditto)
%\documentclass[smallextended]{svjour3}      % onecolumn (second format)
%\documentclass[twocolumn]{svjour3}          % twocolumn


\smartqed  % flush right qed marks, e.g. at end of proof

\usepackage{amssymb}
\usepackage{amsmath}
\usepackage[hmargin=3.4cm,vmargin=3cm]{geometry}
\usepackage{graphicx}
\usepackage[retainorgcmds]{IEEEtrantools}
\usepackage{hyperref}
\hypersetup{
    colorlinks = true,
    urlcolor = {black}
}
\usepackage{rotating}
\usepackage{booktabs}
\usepackage{siunitx}
\usepackage{pdflscape}
% disable labelindent
\let\labelindent\relax
\usepackage{enumitem}
\usepackage{multirow}
\usepackage[normalem]{ulem}
\useunder{\uline}{\ul}{}
\usepackage{setspace} % \doublespacing
\usepackage[utf8]{inputenc}
% \usepackage[numbers,square]{natbib}
% \usepackage{algorithm, algorithmic}
\usepackage[linesnumbered, ruled, vlined]{algorithm2e}
\usepackage{rotating}
%

\usepackage{tkz-graph}

\usepackage{verbatim}
\usetikzlibrary{arrows,arrows.meta,shapes,decorations.pathmorphing, decorations.markings}

\begin{document}

\title{Article title \thanks{This work was partially supported by the Brazilian National Council for Scientific and Technological Development (CNPq), the Foundation for Support of Research of the State of Minas Gerais, Brazil (FAPEMIG), and Coordination for the Improvement of Higher Education Personnel, Brazil (CAPES).}
}
%\subtitle{Do you have a subtitle?\\ If so, write it here}

%\titlerunning{Short form of title}        % if too long for running head

\author{Iago A. Carvalho \and Homer J. Simson}

%\authorrunning{Short form of author list} % if too long for running head

\institute{Iago A. Carvalho \at
            Department of Computer Science, Universidade Federal de Minas Gerais, Av. Antônio Carlos 6627, Belo Horizonte, MG 31270-010, Brazil \\
            Tel.: +55-31-971216117\\
           \email{\href{mailto:iagoac@dcc.ufmg.br}{iagoac@dcc.ufmg.br}} \\            
		   \and           
            Homer J. Simpson \at 
            Fox
}

\date{Received: xxxx / Accepted: yyyy}

\maketitle

\begin{abstract}
There will be an abstract here.
\keywords{Keyword 1 \and Keyword 2 \and Keyword 3 \and Keyword 4 \and Keyword 5}
% \PACS{PACS code1 \and PACS code2 \and more}
% \subclass{MSC code1 \and MSC code2 \and more}
\end{abstract}

\section{Introduction} \label{sec:intro}




\section{Conclusions} \label{sec:conclusions}



% BibTeX users please use one of
%\bibliographystyle{spbasic}      % basic style, author-year citations
\bibliographystyle{spmpsci}      % mathematics and physical sciences
%\bibliographystyle{spphys}       % APS-like style for physics
\bibliography{bibfile}   % name your BibTeX data base

\end{document}
% end of file template.tex