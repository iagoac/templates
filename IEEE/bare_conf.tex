\documentclass[conference]{IEEEtran}

\usepackage[T1]{fontenc}
\usepackage[utf8]{inputenc}
\usepackage[english]{babel}

\usepackage{amssymb}
\usepackage{amsmath}
\usepackage{nicefrac}

\usepackage{graphicx}
\usepackage{subfigure}
\usepackage{float}
\usepackage{xcolor}

\usepackage[linesnumbered, ruled, vlined]{algorithm2e}

\usepackage{rotating}
\usepackage{booktabs}
\usepackage{siunitx}
\usepackage{pdflscape}

\usepackage{hyperref}
\hypersetup{
    colorlinks = true,
    urlcolor = {black}
}
\usepackage{csquotes, lipsum, balance}


\usepackage{rotating, lscape, longtable, tabu, booktabs, siunitx, multirow}
\usepackage{cleveref}
\usepackage{multirow}
\usepackage[normalem]{ulem}
\useunder{\uline}{\ul}{}

% disable labelindent
\let\labelindent\relax
\usepackage{enumitem}

\usepackage{setspace} % \doublespacing

\newcommand{\ie}{i.\,e.}
\newcommand{\eg}{e.\,g.}

\hyphenation{op-tical net-works semi-conduc-tor}

\begin{document}


\title{Article Title}


% author names and affiliations
% use a multiple column layout for up to three different
% affiliations
\author{\IEEEauthorblockN{Iago A. Carvalho}
\IEEEauthorblockA{Computer Science Department\\
Universidade Federal de Minas Gerais\\
Belo Horizonte, Brazil 31270-901\\
Email: \href{mailto:iagoac@dcc.ufmg.br}{iagoac@dcc.ufmg.br}}
\and
\IEEEauthorblockN{Homer Simpson}
\IEEEauthorblockA{Twentieth Century Fox\\
Springfield, USA\\
Email: homer@thesimpsons.com}}


\maketitle

% As a general rule, do not put math, special symbols or citations
% in the abstract
\begin{abstract}
There will be an abstract here
\end{abstract}

% no keywords


\IEEEpeerreviewmaketitle

\section{Introduction} \label{sec:intro}

\section*{Acknowledgment}
This work was partially supported by 
the Brazilian National Council for Scientific and Technological Development (CNPq), 
the Foundation for Support of Research of the State of Minas Gerais, Brazil (FAPEMIG), and 
Coordination for the Improvement of Higher Education Personnel, Brazil (CAPES).

\bibliographystyle{IEEEtran}
\bibliography{bibfile}
% that's all folks
\end{document}


